\documentclass[12pt, a4paper, oneside]{ctexart}
\usepackage{amsmath, amsthm, amssymb, appendix, bm, graphicx, hyperref, mathrsfs}

\CTEXsetup[format={\Large\bfseries}]{section}

\title{\textbf{测试报告}}
\author{第25组}
\date{\today}

\begin{document}

\maketitle

\section{概述}
本次测试中,执行了所有设计的测试用例。测试用例所有执行结果均以截图的形式保存在附件中。

\section{对各功能测试的结果}
\subsection{十六进制转换功能的测试结果}
在对该功能进行测试的过程中,通过了除用例5以外的所有测试用例。

在执行用例5时,输出的结果与预期不符,但是与用例6完全一致。后面又设计了两个略低于最大值的输入(0x3ffffffffffffff3, 0x3fffffffffffffe7),发现输出依然未变。
经过对源代码的分析,发现该问题是由浮点运算损失精度导致。

\section{覆盖率分析}
\subsection{需求覆盖率}
由于仅仅是对五个功能进行用例设计和测试,所以也只覆盖了对应的五个功能的需求。

\subsection{代码覆盖率}
基于同样的原因,本次测试仅覆盖了实现五个功能的javascript代码,未覆盖网页其他代码。

\section{结论和建议措施}
五个功能均能正确实现小整数进制转换。但是当输入变大时,会由于浮点数精度损失导致结果错误。建议改用整数类型进行计算,即可避免该问题。

\end{document}