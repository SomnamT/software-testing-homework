\documentclass[12pt, a4paper, oneside]{ctexart}
\usepackage{amsmath, amsthm, amssymb, appendix, bm, graphicx, hyperref, mathrsfs}

\CTEXsetup[format={\Large\bfseries}]{section}

\title{\textbf{测试计划}}
\author{第25组}
\date{\today}

\begin{document}


\maketitle
\section{测试需求}
本次作业要求使用黑盒测试,使用边界值、等价类、决策表的方法设计测试用例,根据结果定位软件缺陷。

我们小组选择的测试对象是在线进制转换工具(\url{https://www.sojson.com/hexconvert.html}),
该webapp有七个功能,分别是二进制转化、四进制转换、八进制转换、十进制转换、十六进制转换、三十二进制转换、六十四进制转换。
考虑到在实际应用中,三十二进制和六十四进制不常用,且不同的定义会影响这两种进制的表示方法。因此我们决定对前五个功能进行测试。

\section{任务分配}
本小组为四人小组,任务分配为:谭博仁负责测试十六进制转换和结果分析、报告撰写,周峰负责测试八进制转换和十进制转换,
张怡天负责测试二进制转换和四进制转换。


\section{对十六进制转换功能的测试}

\subsection{测试策略分析}

在该功能中,输入为一个独立的、位于[0,0x4000000000000000]的十六进制整数。因此,采用边界值分析和等价类测试是较好的方法。

\subsection{测试用例}

\subsubsection{边界值分析}

边界值分析采用健壮性测试,即考虑无效值的情况。测试用例设计如下(下面所有输入输出均为正则表达式):
\begin{table}[!h]
    \small
    \begin{tabular}{|l|l|l|l|l|l|l|}
    \hline
    用例编号 & 输入 & 二进制 & 四进制 & 八进制 & 十进制 & 十六进制\\ \hline
    1 & -1 & UB & UB & UB & UB & UB \\ \hline
    2 & 0 & 0 & 0 & 0 & 0 & 0\\ \hline
    3 & 1 & 1 & 1 & 1 & 1 & 1 \\ \hline
    4 & 233 & 1000110011 & 20303 & 1063 & 563 & 233 \\ \hline
    5 & 3f\{15\} & 1\{62\} & 3\{31\} & 37\{20\} & 4611686018427388063 & 3f\{15\}\\ \hline
    6 & 40\{15\} & 10\{62\} & 10\{31\} & 40\{20\} & 4611686018427388064 & 40\{15\}\\ \hline
    7 & 40\{14\}1 & UB & UB & UB & UB & UB \\ \hline
    8 & 1.1 & UB & UB & UB & UB & UB\\ \hline
    9 & fg & UB & UB & UB & UB & UB\\ \hline
    \end{tabular}
\end{table}
其中UB表示undefined behaviour,出现UB时输出可以为任意值,但是整个网页不能出现崩溃、卡死等情况。

\newpage
上述用例的说明如下:
\begin{table}[!h]
    \begin{tabular}{|l|l|}
    \hline
    用例编号 & 说明\\ \hline
    1 & 输入略低于最小值\\ \hline
    2 & 输入等于最小值\\ \hline
    3 & 输入略高于最小值 \\ \hline
    4 & 输入为正常值 \\ \hline   
    5 & 输入略低于最大值\\ \hline
    6 & 输入等于最大值 \\ \hline
    7 & 输入略高于最大值 \\ \hline
    8 & 输入为小数 \\ \hline
    9 & 输入不是十六进制 \\ \hline
    \end{tabular}
\end{table}

\subsubsection{等价类分析}

参考我们一般情况下做进制转换的算法,可以将输入按照模16的余数(也就是输入的最后一位)进行分类,因为\textbf{进制转换时对每一位进行的计算过程应当是一致的}。据此可以设计如下测试用例:
\newpage

\begin{table}[!h]
    \small
    \begin{tabular}{|l|l|l|l|l|l|l|}
    \hline
    用例编号 & 输入 & 二进制 & 四进制 & 八进制 & 十进制 & 十六进制\\ \hline
    10 & 40123456789abcdef & UB & UB & UB & UB & UB \\ \hline
    11 & -5 & UB & UB & UB & UB & UB \\ \hline
    12 & 510 & 10100010000 & 110100 & 2420 & 1296 & 510 \\ \hline
    13 & f1 & 11110001 & 3301 & 361 & 241 & f1 \\ \hline
    14 & a2 & 10100010 & 2202 & 242 & 162 & a2 \\ \hline
    15 & 903 & 100100000011 & 210003 & 4403 & 2307 & 903 \\ \hline
    16 & 5b4 & 10110110100 & 112310 & 2664 & 1460 & 5b4 \\ \hline
    17 & 2a5 & 1010100101 & 22211 & 1245 & 677 & 2a5  \\ \hline
    18 & 446 & 10001000110 & 101012 & 2106 & 1094 & 446  \\ \hline
    19 & 877 & 100001110111 & 201313 & 4167 & 2167 & 877  \\ \hline
    20 & 658 & 11001011000 & 121120 & 3130 & 1624 & 658  \\ \hline
    21 & aa9 & 101010101001 & 222221 & 5251 & 2729 & aa9  \\ \hline
    22 & 7da & 11111011010 & 133122 & 3732 & 2010 & 7da  \\ \hline
    23 & 4fb & 10011111011 & 103323 & 2373 & 1275 & 4fb  \\ \hline
    24 & 9fc & 100111111100 & 213330 & 4774 & 2556 & 9fc  \\ \hline
    25 & a3d & 101000111101 & 220331 & 5075 & 2621 & a3d  \\ \hline
    26 & 41e & 10000011110 & 100132 & 2036 & 1054 & 41e  \\ \hline
    27 & 6af & 11010101111 & 122233 & 3257 & 1711 & 6af  \\ \hline
    \end{tabular}
\end{table}
其中,用例10和11分别对应于高于最大值和低于最小值的输入。用例12~17对应于按照模16的余数划分的16个不同等价类的输入。


\end{document}
